\documentclass[a4paper, 12pt, notitlepage]{article}
\usepackage[brazil]{babel}
\usepackage[utf8]{inputenc}
\usepackage[hmargin=2cm,vmargin=3cm,bmargin=3cm]{geometry}
\usepackage{enumerate}
\usepackage{graphicx}
\usepackage{mathtools}
\usepackage{physics}
\usepackage{amsmath,amssymb,amsthm}  %pacotes para matemática, opções de indentação, e links
\usepackage{caption}  % caption to minipages
\usepackage{indentfirst}
\usepackage{makeidx}
\usepackage{hyperref}
\hypersetup{colorlinks=false}
\usepackage[T1]{fontenc}
\usepackage{microtype}

\usepackage[sc,osf]{mathpazo}   % With old-style figures and real smallcaps.
\linespread{1.030}              % Palatino leads a little more leading

% Euler for math and numbers
\usepackage[euler-digits,small]{eulervm}
\AtBeginDocument{\renewcommand{\hbar}{\hslash}}

% Latex plots and drawings
\usepackage{tikz}
\usetikzlibrary{arrows.meta, angles, quotes}
\tikzset{>={Latex[width=3mm,length=3mm]}}  %setas mais visíveis no tikz
\usepackage{pgfplots}
\usepgfplotslibrary{fillbetween}

% some useful math shortcuts
\newtheorem{lema}{Lema }
\newtheorem{teorema}{Teorema }
\newtheorem{corolario}[teorema]{Corolário }
\newtheorem{definicao}{Definição }[section]
\newtheorem{postulado}{Postulado }[section]
\newtheorem{proposicao}{Proposição }[section]
\newtheorem{problema}{Problema }
\newcommand{\cart}{\times}
\newcommand{\ses}{\Longleftrightarrow}
\newcommand{\entao}{\Longrightarrow}
\newcommand{\e}{\wedge}
\newcommand{\ou}{\vee}
\newcommand{\vazio}{\varnothing}
\newcommand{\sobre}{\longrightarrow}
\newcommand{\N}{\mathbb{N}}
\newcommand{\Q}{\mathbb{Q}}
\newcommand{\R}{\mathbb{R}}
\newcommand{\Z}{\mathbb{Z}}
\newcommand{\La}{\mathcal{L}}
\newcommand{\cmod}[3]{#1 \equiv #2\textrm{ (mod }#3\textrm{)}}
\newcommand{\tq}{\textrm{ tal que }}
\renewcommand{\qedsymbol}{$\blacksquare$}
\newcommand{\dsum}{\displaystyle \sum}
\newcommand{\divg}[1]{\vec{\nabla} \cdot #1}
\newcommand{\rot}[1]{\vec{\nabla} \times #1}
\newcommand{\vecb}[1]{\mathbf{ #1}}
\newcommand{\veb}[1]{\mathbf{\hat{#1}}}


\begin{document}
\title{Resolução da Lista 3 de Mecânica Quântica I\\ (F689, Turma B)}
\author{Pedro Rangel Caetano\footnote{Email: p.r.caetano@gmail.com}} 
\date{Universidade Estadual de Campinas, 1o. semestre de 2017}
\maketitle

\tableofcontents
\pagebreak

\begin{enumerate}

% Exercício 1
\addcontentsline{toc}{section}{Exercício 1}
(Bransdeen and Joachian, página 260)\linebreak
Seja o conjunto de operadores: são operadores lineares? e caso sejam são operadores hermitianos?
\begin{enumerate}[(a)]
  \item $\hat{A}_1\Psi(x) = \left(\Psi(x)\right)^2$
  \item $\hat{A}_2\Psi(x) = \frac{d}{dx} \Psi(x)$
  \item $\hat{A}_3\Psi(x) = \int_0^x \Psi(x')dx'$
  \item $\hat{A}_4\Psi(x) = x^2\Psi(x)$
  \item $\hat{A}_5\Psi(x) = \sin \Psi(x)$
  \item $\hat{A}_6\Psi(x) = \frac{d^2}{dx^2} \Psi(x)$
  \item $\hat{A}_7\Psi(x) = -i\hbar\frac{d}{dx}\Psi(x)$
  \item $\hat{A}_8 = 
  \begin{pmatrix} 
    1 && 1\\
    0 && 1\\
  \end{pmatrix}$
  \item $\hat{A}_8 = 
  \begin{pmatrix} 
    1 && i\\
    -i && 1\\
  \end{pmatrix}$
  \item $\hat{\vec{L}} = \hat{\vec{R}} \times \hat{\vec{P}}$, onde $\hat{\vec{R}}$ é o operador vetor posição e $\hat{\vec{P}}$ é o operador vetor momento.
\end{enumerate}

% Exercício 2
\addcontentsline{toc}{section}{Exercício 2}
Mostre que a ação de dois operadores $\hat{A}$ e $\hat{B}$ pode ser representada em forma matricial da seguinte forma: $(AB)_{mn} = \sum_p A_{mp}B_{pn}$.

% Exercício 3
\addcontentsline{toc}{section}{Exercício 3}
Seja o projetor $\hat{P}_n = \ket{\phi_n}\bra{\phi_n}$, onde $\ket{\phi_n}$ são os vetores normalizados de uma base no espaço de Hilbert.
  \begin{enumerate}[(A)]
    \item Mostre que é um operador hermitiano.
    \item O nome projetor vem do fato da propriedade $\hat{P}_n^2 = \hat{P}_n$. Mostre essa propriedade.
    \item Calcule os autovalores e autovetores.
  \end{enumerate}

% Exercício 4
\addcontentsline{toc}{section}{Exercício 4}
Um Hamiltoniano é dado por

\begin{equation}\label{eq:H.neutrinos}
  H = c^2
    \begin{pmatrix}
    m_\mu && m \\
    m && m_\tau
    \end{pmatrix}
\end{equation}

\noindent onde $m$, $m_\mu$ e $m_\tau$ são números reais e os vetores da base são dados por

\begin{equation*}
  \ket{v_\mu} = \begin{pmatrix} 1 \\ 0\end{pmatrix} \qquad
  \ket{v_\tau} = \begin{pmatrix} 0 \\ 1\end{pmatrix}
\end{equation*}

\begin{enumerate}[(A)]
  \item Ache os autovalores e autovetores deste Hamiltoniano.
  \item Assuma que no instante $t=0$, o sistema está no estado $\ket{\Psi(t = 0)} = \ket{v_\mu}$. Então o sistema no instante $t$ estará no estado $\ket{\Psi(t)}$, determine este estado. Qual é a probabilidade de o sistema estar no estado $\ket{v_\tau}$ no instante $t$?\linebreak
  Esta probabilidade está relacionado com o Prêmio Nobel de 2015, pela descoberta da oscilação dos neutrinos. \href{http://assinaturadigital.cienciahoje.org.br/revistas/reduzidas/332/files/assets/basic-html/index.html#1}{Ciência Hoje de Dezembro de 2015: Metamorfose Fantasmagórica}
  \item A matrix $H$ \eqref{eq:H.neutrinos} é Hermitiana? Se sim use a propriedade que pode ser diagonalizada por uma matriz unitária escrita na forma:
  \begin{equation}
  U = 
    \begin{pmatrix}
    \cos \theta && \sin \theta \\
    -\sin \theta && \cos \theta \\
    \end{pmatrix}
  \end{equation}
  Mostre que esta matriz é unitária: $U^{-1} = U^\dagger$.
  Diagonalize a matriz $H$ por esta transformação unitária e ache o valor do ângulo $\theta$ que diagonaliza esta matriz $H$. Como podemos achar os autovetores de $H$ usando este procedimento?
\end{enumerate}

% Exercício 5
\addcontentsline{toc}{section}{Exercício 5}
O Hamiltoniano de um sistema de três níveis é representado pela matriz  
\begin{equation*}
\hat{H} = \hbar \omega
  \begin{pmatrix}
  1 && 0 && 0 \\
  0 && 2 && 0 \\
  0 && 0 && 2 \\
  \end{pmatrix}
\end{equation*}
\noindent e tem dois observáveis $A$ e $B$ representados por
\begin{equation}
\hat{A} = \lambda
  \begin{pmatrix*}
    0 && 1 && 0\\
    1 && 0 && 0\\
    0 && 0 && 2\\
  \end{pmatrix}
\end{equation*}
\noindent e
\begin{equation}
\hat{B} = \mu
  \begin{pmatrix*}
    2 && 0 && 0\\
    0 && 0 && 1\\
    0 && 1 && 0\\
  \end{pmatrix}
\end{equation*}
\noindent onde $\omega$, $\lambda$ e $\mu$ são reais positivos.
\begin{enumerate}[(A)]
  \item Os operadores $A$ e $B$ são operadores lineares? São hermitianos?
  \item Encontre os autovalores e autovetores normalizados de $H$, $A$ e $B$.
  \item Quais são os valores possíveis das quantidades $H$, $A$ e $B$?
  \item Ache os comutadores entre $H$, $A$ e $B$.
  \item Suponha que o sistema começa no estado
    \begin{equation*}
      \ket{\psi(t = 0)} = \begin{pmatrix} 1 \\ 0 \\ 0 \end{pmatrix}
    \end{equation*}
  \noindent que é um estado normalizado. Encontre os valores esperados de $H$, $A$ e $B$ em $t$.
  \item Qual é o estado $\ket{\psi(t)}$? Se você medir a energia no tempo $t$ que valores você pode ter? Qual é a probabilidade de obter cada um desses valores?
\end{enumerate}

%FIM DOS EXERCÍCIOS
\end{enumerate}
\end{document}
