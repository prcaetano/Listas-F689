\documentclass[a4paper, 12pt, notitlepage]{article}
\usepackage[brazil]{babel}
\usepackage[utf8]{inputenc}
\usepackage[hmargin=2cm,vmargin=3cm,bmargin=3cm]{geometry}
\usepackage{enumerate}
\usepackage{graphicx}
\usepackage{mathtools}
\usepackage{physics}
\usepackage{amsmath,amssymb,amsthm}  %pacotes para matemática, opções de indentação, e links
\usepackage{caption}  % caption to minipages
\usepackage{indentfirst}
\usepackage{makeidx}
\usepackage{hyperref}
\hypersetup{colorlinks=false}
\usepackage[T1]{fontenc}
\usepackage{microtype}

\usepackage[sc,osf]{mathpazo}   % With old-style figures and real smallcaps.
\linespread{1.030}              % Palatino leads a little more leading

% Euler for math and numbers
\usepackage[euler-digits,small]{eulervm}
\AtBeginDocument{\renewcommand{\hbar}{\hslash}}

% Latex plots and drawings
\usepackage{tikz}
\usetikzlibrary{arrows.meta, angles, quotes}
\tikzset{>={Latex[width=3mm,length=3mm]}}  %setas mais visíveis no tikz
\usepackage{pgfplots}
\usepgfplotslibrary{fillbetween}

% some useful math shortcuts
\newtheorem{lema}{Lema }
\newtheorem{teorema}{Teorema }
\newtheorem{corolario}[teorema]{Corolário }
\newtheorem{definicao}{Definição }[section]
\newtheorem{postulado}{Postulado }[section]
\newtheorem{proposicao}{Proposição }[section]
\newtheorem{problema}{Problema }
\newcommand{\cart}{\times}
\newcommand{\ses}{\Longleftrightarrow}
\newcommand{\entao}{\Longrightarrow}
\newcommand{\e}{\wedge}
\newcommand{\ou}{\vee}
\newcommand{\vazio}{\varnothing}
\newcommand{\sobre}{\longrightarrow}
\newcommand{\N}{\mathbb{N}}
\newcommand{\Q}{\mathbb{Q}}
\newcommand{\R}{\mathbb{R}}
\newcommand{\Z}{\mathbb{Z}}
\newcommand{\La}{\mathcal{L}}
\newcommand{\cmod}[3]{#1 \equiv #2\textrm{ (mod }#3\textrm{)}}
\newcommand{\tq}{\textrm{ tal que }}
\renewcommand{\qedsymbol}{$\blacksquare$}
\newcommand{\dsum}{\displaystyle \sum}
\newcommand{\divg}[1]{\vec{\nabla} \cdot #1}
\newcommand{\rot}[1]{\vec{\nabla} \times #1}
\newcommand{\vecb}[1]{\mathbf{ #1}}
\newcommand{\veb}[1]{\mathbf{\hat{#1}}}


\begin{document}
\title{Resolução da Lista 6 de Mecânica Quântica I\\ (F689, Turma B)}
\author{Pedro Rangel Caetano\footnote{Email: p.r.caetano@gmail.com}} 
\date{Universidade Estadual de Campinas, 1o. semestre de 2017}
\maketitle

\tableofcontents
\pagebreak


\begin{enumerate}
% Exercício 1
\addcontentsline{toc}{section}{Exercício 1}
\item Dado a equação de autovalores para spin $s=1$, ache a representação matricial de $S^2$, $S_z$, $S_x$ e $S_y$.

\begin{equation}
S^2 \ket{s\,m_s} = s(s+1)\hbar^2\ket{s\,m_s} \qquad
S_z \ket{s\,m_s} = m_s \hbar \ket{s\,m_s}
\end{equation}

\noindent siga o raciocínio feito na aula do dia 12 de junho e descrito nas notas de aula 22 a 24 que estão no link do curso nas páginas 217 a 219. A forma matricial de $S_x$ é igual à de $L_x$ dada no exercício 4 da Lista 5.

\textbf{Resolução: }

Quando $s = 1$, $m_s$ pode assumir os valores $1$, $0$ e $-1$. Há portanto três autoestados simultâneos de $S^2$ e $S_z$:

\[
  \ket{1} \equiv \ket{1\,1} \qquad
  \ket{0} \equiv \ket{1\,0} \qquad
  \ket{-1} \equiv \ket{1\,-1}
\]

Lembremos agora como obter a representação de um operador $A$ numa base ortonormal $\ket{e_i}$ de um espaço vetorial: se $\ket{v} = \sum v_i \ket{e_i}$ e $\ket{w} = A\ket{v} = \sum w_i \ket{e_i}$ temos

\begin{align*}
  w_i &= \left\langle e_i | w\right\rangle \\
  &= \bra{e_i} A \ket{v} \\
  &= \sum_j \bra{e_i} A \ket{e_j} v_j \\
  &= \sum_j A_{ij} v_j
\end{align*}

\noindent ou seja, o elemento $A_{ij}$ da representação matricial de $A$ é dado por

\[ A_{ij} = \bra{e_i} A \ket{e_j} \]

Começaremos então escrevendo a representação matricial de $S^2$. Primeiramente, calculando $S^2$ nos vetores da base temos

\[
S^2 \ket{1} = 2 \hbar^2 \ket{1} \qquad S^2 \ket{0} = 2\hbar^2 \ket{0} \qquad S^2 \ket{-1} = 2\hbar^2 \ket{-1}
\]

A representação matricial de $S^2$ é portanto (lembrando que a base $\{\ket{1}, \ket{0}, \ket{-1}\}$ é ortonormal)

\[
S^2 = \begin{pmatrix} 
  \bra{1} S^2 \ket{1} & \bra{1} S^2 \ket{0} & \bra{1} S^2 \ket{-1} \\
  \bra{0} S^2 \ket{1} & \bra{0} S^2 \ket{0} & \bra{0} S^2 \ket{-1} \\
  \bra{-1} S^2 \ket{1} & \bra{-1} S^2 \ket{0} & \bra{-1} S^2 \ket{-1}
\end{pmatrix}
=
\begin{pmatrix}
  2 \hbar^2 & 0 & 0 \\
  0 & 2\hbar^2 & 0 \\
  0 & 0 & 2\hbar^2
\end{pmatrix}
\]

\noindent o que já era esperado pois a base que utilizamos é, por definição, a base comum de $S^2$ e $S_z$: $S^2$ e $S_z$ devem portanto ser diagonais. Para $S_z$ portanto obtemos

\[ 
S_z\ket{1} = \hbar \ket{1} \qquad S_z \ket{0} = \hbar \ket{0} \qquad S_z \ket{-1} = -\hbar \ket{-1} 
\]

\noindent logo

\[
S_z = \begin{pmatrix}
  \bra{1} S_z \ket{1} & \bra{1} S_z \ket{0} & \bra{1} S_z \ket{-1} \\
  \bra{0} S_z \ket{1} & \bra{0} S_z \ket{0} & \bra{0} S_z \ket{-1} \\
  \bra{-1} S_z \ket{1} & \bra{-1} S_z \ket{0} & \bra{-1} S_z \ket{-1}
\end{pmatrix}
= 
\begin{pmatrix}
  \hbar & 0 & 0 \\
  0 & 0 & 0 \\
  0 & 0 & -\hbar
\end{pmatrix}
\]

Para calcular $S_x$ e $S_y$, o truque é lembrar que os operadores de levantamento/abaixamento são escritos

\[ S_+ = S_x + iS_y \qquad S_- = S_x - iS_y \]

Portanto

\[ S_x = \frac{S_+ + S_-}{2} \qquad S_y = \frac{S_+ - S_-}{2i}\]

Basta portanto obter as representações matriciais de $S_+$ e $S_-$ e conseguiremos as representações de $S_x$ e $S_y$. Como

\[ S_\pm \ket{s\,m_s} = \sqrt{s(s+1) - m_s(m_s \pm 1)}\hbar\ket{s\,m_s\pm 1} \]

\noindent temos

\begin{align*}
  S_+ \ket{1} = 0 \qquad S_+ \ket{0} = \sqrt{2}\hbar\ket{1} \qquad S_+ \ket{-1} = \sqrt{2} \hbar \ket{-1} \\
  S_- \ket{1} = \sqrt{2} \ket{0} \qquad S_- \ket{0} = \sqrt{2} \ket{-1} \qquad S_- \ket{-1} = 0
\end{align*}

\noindent e, portanto

\[
  S_+ = \begin{pmatrix} 
  \bra{1} S_+ \ket{1} & \bra{1} S_+ \ket{0} & \bra{1} S_+ \ket{-1} \\
  \bra{0} S_+ \ket{1} & \bra{0} S_+ \ket{0} & \bra{0} S_+ \ket{-1} \\
  \bra{-1} S_+ \ket{1} & \bra{-1} S_+ \ket{0} & \bra{-1} S_+ \ket{-1}
\end{pmatrix}
=
\begin{pmatrix}
  0 & \sqrt{2} \hbar & 0 \\
  0 & 0 & \sqrt{2} \hbar \\
  0 & 0 & 0
\end{pmatrix}
\]

\[
  S_- = \begin{pmatrix} 
  \bra{1} S_- \ket{1} & \bra{1} S_- \ket{0} & \bra{1} S_- \ket{-1} \\
  \bra{0} S_- \ket{1} & \bra{0} S_- \ket{0} & \bra{0} S_- \ket{-1} \\
  \bra{-1} S_- \ket{1} & \bra{-1} S_- \ket{0} & \bra{-1} S_- \ket{-1}
\end{pmatrix}
=
\begin{pmatrix}
  0 & 0 & 0 \\
  \sqrt{2}\hbar & 0 & 0 \\
  0 & \sqrt{2}\hbar & 0
\end{pmatrix}
\]

Note duas coisas: primeiramente, só precisamos calcular uma dentre $S_+$ e $S_-$ para obter a outra, pois, como $S_- = S_+^\dagger$, podemos obter a representação matricial de uma a partir da outra simplemente transpondo e conjugando. Ademais, note que $S_+$ possui apenas a diagonal imediatamente acima da diagonal principal preenchida (e $S_-$ a diagonal imediatamente abaixo). Isto sempre acontece com os operadores levantamento e abaixamento (afinal, eles são operadores de levantamento e abaixamento!).

Agora, é imediato obter

\[
S_x = \frac{S_+ + S_-}{2} = \begin{pmatrix}
  0 & \frac{\hbar}{\sqrt{2}} & 0 \\
  \frac{\hbar}{\sqrt{2}} & 0 & \frac{\hbar}{\sqrt{2}} \\
  0 & \frac{\hbar}{\sqrt{2}} & 0
\end{pmatrix}
\]

\noindent e

\[
S_y = \frac{S_+ - S_-}{2i} = \begin{pmatrix}
  0 & \frac{-i\hbar}{\sqrt{2}} & 0 \\
  \frac{i\hbar}{\sqrt{2}} & 0 & \frac{-i\hbar}{\sqrt{2}} \\
  0 & \frac{i\hbar}{\sqrt{2}} & 0
\end{pmatrix}
\]

Note que $S_x$ e $S_y$ são hermiteanos, como necessariamente deveriam ser!


% Exercício 2
\addcontentsline{toc}{section}{Exercício 2}
\item Cohen, página 476, Complemento $J_{IV}$, exercício 1.\newline
\textit{Precessão de Larmor}\newline
Considere uma partícula de spin $1/2$ de momento magnético $\vec{M} = \gamma \vec{S}$. Os estados de spin são descritos na base de vetores $\ket{+}$ e $\ket{-}$, autovetores de $S_z$ com autovalores $\hbar/2$ e $-\hbar/2$. No tempo $t=0$, o estado do sistema é

\begin{equation}
  \ket{\Psi(t=0)} = \ket{+}
\label{eq:psi.t0}
\end{equation}

\textit{Observação: na notação do Cohen $\ket{+}$ é o estado de spin up na direção z, ou sendo mais preciso, $\ket{1/2\,1/2}_z$.}\newline
\textit{Observação: na notação do Griffiths, $\gamma = \frac{qg_s}{2m}$, onde $q$ é a carga, $g_s$ é o fator giromagnético e $m$ é a massa da partícula.}

\begin{enumerate}[(a)]
 \item Se o observável $S_x$ é medido no tempo $t=0$, quais resultados são possíveis e com quais probabilidades?
 \item Em vez de medir o resultado no tempo $t=0$, o sistema é deixado evoluir sob a influência de um campo magnético na direção $y$, de módulo $B_0$. Calcule na base $\ket{+}$, $\ket{-}$, o estado do sistema no instante $t$.
 \item No tempo $t$, nós medimos os observáveis $S_x$, $S_y$ e $S_z$. Quais são os valores destas quantidades que podemos encontrar e com quais probabilidades?
 \item Qual é a relação entre $B_0$ e $t$ para qual o resultado de uma das medidas é sem incerteza? Dê uma interpretação física desta condição.
\end{enumerate}

\textbf{Resolução:}
\begin{enumerate}[(a)]
  \item Como de costume, os valores possíveis são os autovalores do operador $S_x$ e as respectivas probabilidades são obtidas pela projeção nos autovetores. Já esperamos que os autovalores sejam $\hbar/2$ e $-\hbar/2$, como os de $S_z$, mas podemos checar isto: o operador $S_x$ é representado na base $\{\ket{+}, \ket{-}\}$ como
  
\[
  S_x = \frac{\hbar}{2} \sigma_x = \frac{\hbar}{2}\begin{pmatrix}
  0 & 1 \\
  1 & 0
  \end{pmatrix}
\]

Logo os autovalores são dados por

\begin{align*} 
  \begin{vmatrix}
   -s_x & \frac{\hbar}{2} \\
   \frac{\hbar}{2} & -s_x
   \end{vmatrix} &= 0 \\
   s_x^2 - \frac{\hbar^2}{4} &= 0 \\
   \therefore s_x &= \pm \frac{\hbar}{2}
\end{align*}

Representando o autovetor associado a $\hbar/2$ como $\ket{+}_x$ (e o autovetor associado a $-\hbar/2$, $\ket{-}_x$) temos

\begin{align*}
  S_x \ket{+}_x &= \frac{\hbar}{2} \ket{+}_x \\
  \frac{\hbar}{2}\begin{pmatrix}
  0 & 1 \\
  1 & 0
  \end{pmatrix}\begin{pmatrix}
  \alpha \\
  \beta
  \end{pmatrix} &= 
  \frac{\hbar}{2}\begin{pmatrix}
  \alpha \\
  \beta
  \end{pmatrix} \\
  \begin{pmatrix} \alpha \\ \beta \end{pmatrix} &=
  \begin{pmatrix} \beta \\ \alpha \end{pmatrix}
\end{align*}

Já impondo a normalização, temos portanto que $\ket{+}_x$ é representado na base de autovetores de $S_z$ como

\begin{align*}
 \ket{+}_x &= \begin{pmatrix} \frac{1}{\sqrt{2}} \\ \frac{1}{\sqrt{2}} \end{pmatrix} \\
  &= \frac{1}{\sqrt{2}} \ket{+} + \frac{1}{\sqrt{2}} \ket{-}
\end{align*}

Procedimento semelhante para $\ket{-}_x$ fornece

\begin{align*}
  \ket{-}_x &= \frac{1}{\sqrt{2}} \ket{+} - \frac{1}{\sqrt{2}} \ket{-}
\end{align*}

\noindent (alternativamente podemos lembrar que, já que $\ket{+}_x$ e $\ket{-}_x$ devem ser ortogonais, basta trocar o sinal de uma das componentes para obter um a partir do outro).

Como em $t = 0$ a partícula esta no estado $\ket{+}$, temos que a probabilidade de obter o valor $\hbar/2$ para uma medida de $S_x$ é

\begin{align*}
  \mathcal{P}\left[ S_x = \hbar/2\right] &= \left|\prescript{}{x}{\left\langle -|+\right\rangle}\right|^2 \\
  &= \left|\left[ \frac{1}{\sqrt{2}} \bra{+} + \frac{1}{\sqrt{2}} \bra{-} \right] \ket{+}\right|^2 \\
  &= \frac{1}{2}
\end{align*}

\noindent e a probabilidade de obter o valor $-\hbar/2$ por sua vez vale

\begin{align*}
  \mathcal{P}\left[ S_x = -\hbar/2\right] &= \left|\prescript{}{x}{\left\langle -|+\right\rangle}\right|^2 \\
  &= \left| \left[\frac{1}{\sqrt{2}}\bra{+} - \frac{1}{\sqrt{2}}\bra{-} \right]\ket{+}\right|^2 \\
  &= \frac{1}{2}
\end{align*}

\item Começamos lembrando que a energia potencial de uma particula com momento magnético $\vec{M}$ num campo magnético $\vec{B}$ é dada por $-\vec{M}\cdot\vec{B}$. A energia potencial neste caso vale, portanto, $-\gamma B_0 S_y$. O Hamiltoniano da partícula é, então\footnote{Onde foi parar a energia cinética? Neste caso a dinâmica dos graus de liberdade ligados à translação é desacoplada da dinâmica dos graus de liberdade de spin. Podemos resolvê-los separadamente, portanto. A situação é de certa forma análoga à separação de variáveis ao resolver uma EDP.}

\[ H = -V = \gamma B_0 S_y \]

Os autoestados de $H$ são então $\ket{+}_y$ e $\ket{-}_y$, os autovetores de $S_y$. Na base destes autovetores, o Hamiltoniano é representado como

\[ H = \frac{\gamma B_0 \hbar}{2} 
  \begin{pmatrix} 
  1 && 0 \\
  0 && -1
  \end{pmatrix} = \begin{pmatrix}
  E_+ & 0 \\
  0 & E_-
  \end{pmatrix}
\]

A vantagem de trabalhar na base de autoestados é que a evolução temporal é simples: se em $t=0$ o sistema esta no estado $a \ket{+}_y + b\ket{-}_y$, no tempo $t$ sabemos que ele estará no estado $ae^{-\frac{iE_+t}{\hbar}} \ket{+}_y + be^{-\frac{iE_-t}{\hbar}} \ket{-}_y$. Para obter o estado do sistema no instante $t$ temos então que

\begin{itemize}
  \item Escrever o estado do sistema em $t=0$ na base de autovetores de $S_y$, a base de autoestados de $H$;
  \item Nesta nova base, calcular o estado do sistema no tempo $t$ e, por fim;
  \item Reescrever o estado do sistema na base original
\end{itemize}

Para realizar essas mudanças de base precisamos calcular os autovetores de $S_y$ na base de $S_z$. Lembrando que a representação de $S_y$ nesta base é

\[ S_y = \frac{\hbar}{2}\sigma_y = \frac{\hbar}{2} \begin{pmatrix} 0 & -i \\ i & 0 \end{pmatrix} \]

\noindent temos que o autovetor $\ket{+}_y$ pode ser calculado por

\begin{align*}
  S_y \ket{+}_y &= \frac{\hbar}{2} \ket{+}_y \\
  \frac{\hbar}{2} \begin{pmatrix} 0 & -i \\ i & 0 \end{pmatrix}  \begin{pmatrix} \alpha \\ \beta \end{pmatrix} &=
  \frac{\hbar}{2} \begin{pmatrix} \alpha \\ \beta \end{pmatrix} \\
  \begin{pmatrix}-i\beta \\ i\alpha \end{pmatrix} &= \begin{pmatrix} \alpha \\ \beta \end{pmatrix}
\end{align*}

\noindent já normalizando o vetor, encontramos então

\begin{align*}
  \ket{+}_y &= \begin{pmatrix} \frac{1}{\sqrt{2}}\\ \frac{i}{\sqrt{2}} \end{pmatrix} \\
  &= \frac{1}{\sqrt{2}} \ket{+} + \frac{i}{\sqrt{2}} \ket{-}
\end{align*}

Procedendo analogamente, ou utilizando a condição de ortogonalidade, temos que $\ket{-}_y$ é dado por

\begin{align*}
  \ket{-}_y &= \begin{pmatrix} \frac{1}{\sqrt{2}}\\ \frac{-i}{\sqrt{2}} \end{pmatrix} \\
  &= \frac{1}{\sqrt{2}} \ket{+} - \frac{i}{\sqrt{2}} \ket{-}
\end{align*}

Para expressar $\ket{+}$ na base de $S_y$, lembremos agora a identidade

\[ 1 = \ket{+}_y \prescript{}{y}{\bra{+}} + \ket{-}_y \prescript{}{y}{\bra{-}} \]

\noindent de onde obtemos

\[ \ket{+} = 1\ket{+} = \prescript{}{y}{\left\langle + | + \right\rangle}\ket{+} + \prescript{}{y}{\left\langle - | + \right\rangle}\ket{-} \]

Portanto, conjugando as expressões para $\ket{+}_y$ e $\ket{-}_y$ obtidas

\begin{align*}
  \ket{+} &= \left[ \frac{1}{\sqrt{2}} \left\langle +|+ \right\rangle + \frac{i}{\sqrt{2}} \left\langle -|+ \right\rangle \right] \ket{+}_y + \left[ \frac{1}{\sqrt{2}} \left\langle +|- \right\rangle - \frac{i}{\sqrt{2}} \left\langle -|- \right\rangle \right] \ket{-}_y \\
  &= \frac{1}{\sqrt{2}} \ket{+}_y + \frac{1}{\sqrt{2}} \ket{-}_y
\end{align*}

Já expressamos $\ket{\Psi(t = 0)}$ na base de $S_y$. Seguindo com nosso plano, podemos agora calcular $\ket{\Psi(t)}$ nesta base

\begin{align*}
  \ket{\Psi(t)} &= \frac{1}{\sqrt{2}} e^{-iE_+ t/\hbar}\ket{+}_y + \frac{1}{\sqrt{2}} e^{-iE_-t/\hbar} \ket{-}_y \\
  &= \frac{1}{\sqrt{2}} e^{-\gamma B_0 t/2} \ket{+}_y + \frac{1}{\sqrt{2}} e^{\gamma B_0 t/2} \ket{-}_y
\end{align*}

Por fim, resta-nos reexpressar este estado na base de $S_z$:

\begin{align*}
  \ket{\Psi(t)} &= \frac{1}{\sqrt{2}} e^{-\gamma B_0 t/2} \left[ \frac{1}{\sqrt{2}} \ket{+} + \frac{i}{\sqrt{2}} \ket{-}\right] + \frac{1}{\sqrt{2}} e^{\gamma B_0 t/2} \left[ \frac{1}{\sqrt{2}} \ket{+} - \frac{i}{\sqrt{2}} \ket{-}\right] \\
  &= \frac{1}{2} \left(e^{-\gamma B_0 t/2} + e^{\gamma B_0 t/2}\right) \ket{+} + \frac{1}{2i}\left(e^{\gamma B_0 t/2} - e^{-\gamma B_0 t/2}\right) \ket{-} \\
  &= \cos \frac{\gamma B_0 t}{2} \ket{+} + \sin \frac{\gamma B_0 t}{2} \ket{-} \\
  &= \cos \omega t \ket{+} + \sin \omega t \ket{-}
\end{align*}

% ALTERNATIVA
\item Os valores que podemos encontrar são os autovalores dos observáveis: $+ \hbar/2$ e $- \hbar/2$. Como já obtemos $\ket{\Psi(t)}$ e as expressões das bases de $S_x$ e $S_y$ em termos da base de $S_z$, tudo o que nos resta são contas corriqueiras. Começando com $S_x$, podemos calcular as probabilidades

\begin{align*}
  \mathcal{P}\left[ S_x = \frac{\hbar}{2}\right] &= \left| \prescript{}{x}{\left\langle +|\Psi(t)\right\rangle} \right|^2 \\
  &= \left| \left[\frac{1}{\sqrt{2}} \bra{+} + \frac{1}{\sqrt{2}}\bra{-}\right] \left[\cos \omega t \ket{+} + \sin \omega t \ket{-} \right] \right |^2 \\
  &= \left|\frac{\cos \omega t}{\sqrt{2}}  + \frac{\sin\omega t}{\sqrt{2}}\right|^2 \\
  &= \frac{1}{2}\left[ 1 + \sin \gamma B_0 t\right]
\end{align*}

\noindent e

\begin{align*}
  \mathcal{P}\left[S_x = -\frac{\hbar}{2}\right] &= 1 - \mathcal{P}\left[S_x = -\frac{\hbar}{2}\right] \\
  &= \frac{1}{2}\left[1 - \sin \gamma B_0 t \right]
\end{align*}

Já para $S_y$:

\begin{align*}
  \mathcal{P}\left[ S_y = \frac{\hbar}{2}\right] &= \left| \prescript{}{y}{\left\langle + | \Psi(t) \right\rangle} \right|^2 \\
  &= \left| \left[\frac{1}{\sqrt{2}} \bra{+} - \frac{i}{\sqrt{2}}\bra{-}\right] \left[\cos \omega t \ket{+} + \sin \omega t \ket{-} \right] \right |^2 \\
  &= \left|\frac{\cos \omega t}{\sqrt{2}} - \frac{i\sin\omega t}{\sqrt{2}}\right|^2 \\
  &= \frac{1}{2}
\end{align*}

\noindent e portanto

\begin{align*}
  \mathcal{P}\left[ S_y = \frac{\hbar}{2} \right] &= \frac{1}{2}
\end{align*}

Por fim, temos para $S_z$:

\begin{align*}
  \mathcal{P} \left[ S_z = \frac{\hbar}{2} \right] &= \left|\left\langle +| \Psi(t) \right\rangle \right|^2 \\
  &= \left| \bra{+} \left[\cos \omega t \ket{+} + \sin \omega t \ket{-} \right] \right|^2 \\
  &= \cos^2 \frac{\gamma B_0 t}{2}
\end{align*}

\noindent e

\begin{align*}
  \mathcal{P} \left[ S_z = -\frac{\hbar}{2} \right] &= \sin^2 \frac{\gamma B_0 t}{2}
\end{align*}

\item Para que o resultado de uma medida seja sem incerteza, a probabilidade de obtê-la deve ser exatamente 1. Das expressões deduzidas no item anterior, notamos que tanto para $S_x$ quanto para $S_z$ esta condição pode ser satisfeita. Para que uma medida de $S_x$ resulte $\hbar / 2$ ou $-\hbar/2$ com certeza, é necessário que

\[ \gamma B_0 t = \frac{\pi}{2} + 2n\pi \qquad \text{ ou } \qquad \gamma B_0 t = \frac{3\pi}{2} + 2n\pi \]

Por outro lado, para que uma medida de $S_z$ resulte $\hbar/2$ ou $-\hbar/2$ com certeza, devemos ter

\[ \gamma B_0 t = 2n\pi \qquad \text{ ou } \qquad \gamma B_0 t = (2n + 1) \pi \]

O que todas estas relações tem em comum é que, a cada intervalo de tempo de $T = \frac{2\pi}{\gamma B_0}$ elas se repetem. Este tempo é justamente o período de precessão. Classicamente, quando aplicamos um campo magnético $\vec{B} = B_0\hat{y}$ a um sistema com momento de dipolo $\vec{M} = \gamma L_z\hat{z}$, este sistema sofre um torque $\vec{\tau} = \vec{M} \times \vec{B} = \gamma B_0 L_z \hat{x} $. Como este torque é perpendicular ao momento angular $L_z$, este momento precessiona ao redor do eixo $y$, com velocidade angular de precessão

\[ \omega_p = \frac{\tau}{L_z} = \gamma B_0 \]

O período de precessão é portanto

\[ T = \frac{2\pi}{\omega_p} = \frac{2\pi}{\gamma B_0} \]

O que coincide exatamente com o período que encontramos acima. Na transição de clássico para quântico, o que acontece é que a maneira correta de interpretar a precessão é diferente: agora pensamos na precessão do valor esperado do observável $\vec{S}$. Ocorre que, no caso que estamos tratando, quando os valores esperados das componentes $S_x$ ou $S_z$ são $\pm \hbar / 2$ não pode haver incerteza (por exemplo, note que se o valor esperado de $S_x$ for $\hbar / 2$, sendo este valor dado por $\left\langle S_x \right\rangle = \mathcal{P}\left[S_x = \hbar/2\right] \frac{\hbar}{2} + \mathcal{P}\left[S_x = -\hbar/2\right]\frac{-\hbar}{2}$ é claro que a probabilidade de obter o resultado $\hbar/2$ deve ser $1$). Para uma discussão mais detalhada, recomendo consultar a seção 4.4.2 do Griffiths.

\end{enumerate}

%FIM DOS EXERCÍCIOS
\end{enumerate}
\end{document}
