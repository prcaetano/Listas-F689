\documentclass[a4paper, 12pt, notitlepage]{article}
\usepackage[brazil]{babel}
\usepackage[utf8]{inputenc}
\usepackage[hmargin=2cm,vmargin=3cm,bmargin=3cm]{geometry}
\usepackage{enumerate}
\usepackage{graphicx}
\usepackage{mathtools}
\usepackage{physics}
\usepackage{amsmath,amssymb,amsthm}  %pacotes para matemática, opções de indentação, e links
\usepackage{caption}  % caption to minipages
\usepackage{indentfirst}
\usepackage{makeidx}
\usepackage{hyperref}
\hypersetup{colorlinks=false}
\usepackage[T1]{fontenc}
\usepackage{microtype}

\usepackage[sc,osf]{mathpazo}   % With old-style figures and real smallcaps.
\linespread{1.030}              % Palatino leads a little more leading

% Euler for math and numbers
\usepackage[euler-digits,small]{eulervm}
\AtBeginDocument{\renewcommand{\hbar}{\hslash}}

% Latex plots and drawings
\usepackage{tikz}
\usetikzlibrary{arrows.meta, angles, quotes}
\tikzset{>={Latex[width=3mm,length=3mm]}}  %setas mais visíveis no tikz
\usepackage{pgfplots}
\usepgfplotslibrary{fillbetween}

% some useful math shortcuts
\newtheorem{lema}{Lema }
\newtheorem{teorema}{Teorema }
\newtheorem{corolario}[teorema]{Corolário }
\newtheorem{definicao}{Definição }[section]
\newtheorem{postulado}{Postulado }[section]
\newtheorem{proposicao}{Proposição }[section]
\newtheorem{problema}{Problema }
\newcommand{\cart}{\times}
\newcommand{\ses}{\Longleftrightarrow}
\newcommand{\entao}{\Longrightarrow}
\newcommand{\e}{\wedge}
\newcommand{\ou}{\vee}
\newcommand{\vazio}{\varnothing}
\newcommand{\sobre}{\longrightarrow}
\newcommand{\N}{\mathbb{N}}
\newcommand{\Q}{\mathbb{Q}}
\newcommand{\R}{\mathbb{R}}
\newcommand{\Z}{\mathbb{Z}}
\newcommand{\La}{\mathcal{L}}
\newcommand{\cmod}[3]{#1 \equiv #2\textrm{ (mod }#3\textrm{)}}
\newcommand{\tq}{\textrm{ tal que }}
\renewcommand{\qedsymbol}{$\blacksquare$}
\newcommand{\dsum}{\displaystyle \sum}
\newcommand{\divg}[1]{\vec{\nabla} \cdot #1}
\newcommand{\rot}[1]{\vec{\nabla} \times #1}
\newcommand{\vecb}[1]{\mathbf{ #1}}
\newcommand{\veb}[1]{\mathbf{\hat{#1}}}


\begin{document}
\title{Resolução da Lista 5 de Mecânica Quântica I\\ (F689, Turma B)}
\author{Pedro Rangel Caetano\footnote{Email: p.r.caetano@gmail.com}} 
\date{Universidade Estadual de Campinas, 1o. semestre de 2017}
\maketitle

\tableofcontents
\pagebreak


\begin{enumerate}
% Exercício 1
\addcontentsline{toc}{section}{Exercício 1}
\item Dado a equação de autovalores para spin $s=1$, ache a representação matricial de $S^2$, $S_z$, $S_x$ e $S_y$.

\begin{equation}
S^2 \ket{s\,m_s} = s(s+1)\hbar^2\ket{s\,m_s} \qquad
S_z \ket{s\,m_s} = m_s \hbar \ket{s\,m_s}
\end{equation}

\noindent siga o raciocínio feito na aula do dia 12 de junho e descrito nas notas de aula 22 a 24 que estão no link do curso nas páginas 217 a 219. A forma matricial de $S_x$ é igual à de $L_x$ dada no exercício 4 da Lista 5.

\textbf{Resolução: }

Quando $s = 1$, $m_s$ pode assumir os valores $1$, $0$ e $-1$. Há portanto três autoestados simultâneos de $S^2$ e $S_z$:

\[
  \ket{1} \equiv \ket{1\,1} \qquad
  \ket{0} \equiv \ket{1\,0} \qquad
  \ket{-1} \equiv \ket{1\,-1}
\]

Lembremos agora como obter a representação de um operador $A$ numa base ortonormal $\ket{e_i}$ de um espaço vetorial: se $\ket{v} = \sum v_i \ket{e_i}$ e $\ket{w} = A\ket{v} = \sum w_i \ket{e_i}$ temos

\begin{align*}
  w_i &= \left\langle e_i | w\right\rangle \\
  &= \bra{e_i} A \ket{v} \\
  &= \sum_j \bra{e_i} A \ket{e_j} v_j \\
  &= \sum_j A_{ij} v_j
\end{align*}

\noindent ou seja, o elemento $A_{ij}$ da representação matricial de $A$ é dado por

\[ A_{ij} = \bra{e_i} A \ket{e_j} \]

Começaremos então escrevendo a representação matricial de $S^2$. Primeiramente, calculando $S^2$ nos vetores da base temos

\[
S^2 \ket{1} = 2 \hbar^2 \ket{1} \qquad S^2 \ket{0} = 2\hbar^2 \ket{0} \qquad S^2 \ket{-1} = 2\hbar^2 \ket{-1}
\]

A representação matricial de $S^2$ é portanto (lembrando que a base $\{\ket{1}, \ket{0}, \ket{-1}\}$ é ortonormal)

\[
S^2 = \begin{pmatrix} 
  \bra{1} S^2 \ket{1} & \bra{1} S^2 \ket{0} & \bra{1} S^2 \ket{-1} \\
  \bra{0} S^2 \ket{1} & \bra{0} S^2 \ket{0} & \bra{0} S^2 \ket{-1} \\
  \bra{-1} S^2 \ket{1} & \bra{-1} S^2 \ket{0} & \bra{-1} S^2 \ket{-1}
\end{pmatrix}
=
\begin{pmatrix}
  2 \hbar^2 & 0 & 0 \\
  0 & 2\hbar^2 & 0 \\
  0 & 0 & 2\hbar^2
\end{pmatrix}
\]

\noindent o que já era esperado pois a base que utilizamos é, por definição, a base comum de $S^2$ e $S_z$: $S^2$ e $S_z$ devem portanto ser diagonais. Para $S_z$ portanto obtemos

\[ 
S_z\ket{1} = \hbar \ket{1} \qquad S_z \ket{0} = \hbar \ket{0} \qquad S_z \ket{-1} = \hbar \ket{-1} 
\]

\noindent logo

\[
S_z = \begin{pmatrix}
  \bra{1} S_z \ket{1} & \bra{1} S_z \ket{0} & \bra{1} S_z \ket{-1} \\
  \bra{0} S_z \ket{1} & \bra{0} S_z \ket{0} & \bra{0} S_z \ket{-1} \\
  \bra{-1} S_z \ket{1} & \bra{-1} S_z \ket{0} & \bra{-1} S_z \ket{-1}
\end{pmatrix}
= 
\begin{pmatrix}
  \hbar & 0 & 0 \\
  0 & 0 & 0 \\
  0 & 0 & \hbar
\end{pmatrix}
\]

Para calcular $S_x$ e $S_y$, o truque é lembrar que os operadores de levantamento/abaixamento são escritos

\[ S_+ = S_x + iS_y \qquad S_- = S_x - iS_y \]

Portanto

\[ S_x = \frac{S_+ + S_-}{2} \qquad S_y = \frac{S_+ - S_-}{2i}\]

Basta portanto obter as representações matriciais de $S_+$ e $S_-$ e conseguiremos as representações de $S_x$ e $S_y$. Como

\[ S_\pm \ket{s\,m_s} = \sqrt{s(s+1) - m_s(m_s \pm 1)}\hbar\ket{s\,m_s\pm 1} \]

\noindent temos

\begin{align*}
  S_+ \ket{1} = 0 \qquad S_+ \ket{0} = \sqrt{2}\hbar\ket{1} \qquad S_+ \ket{-1} = \sqrt{2} \hbar \ket{-1} \\
  S_- \ket{1} = \sqrt{2} \ket{0} \qquad S_- \ket{0} = \sqrt{2} \ket{-1} \qquad S_- \ket{-1} = 0
\end{align*}

\noindent e, portanto

\[
  S_+ = \begin{pmatrix} 
  \bra{1} S_+ \ket{1} & \bra{1} S_+ \ket{0} & \bra{1} S_+ \ket{-1} \\
  \bra{0} S_+ \ket{1} & \bra{0} S_+ \ket{0} & \bra{0} S_+ \ket{-1} \\
  \bra{-1} S_+ \ket{1} & \bra{-1} S_+ \ket{0} & \bra{-1} S_+ \ket{-1}
\end{pmatrix}
=
\begin{pmatrix}
  0 & \sqrt{2} \hbar & 0 \\
  0 & 0 & \sqrt{2} \hbar \\
  0 & 0 & 0
\end{pmatrix}
\]

\[
  S_- = \begin{pmatrix} 
  \bra{1} S_- \ket{1} & \bra{1} S_- \ket{0} & \bra{1} S_- \ket{-1} \\
  \bra{0} S_- \ket{1} & \bra{0} S_- \ket{0} & \bra{0} S_- \ket{-1} \\
  \bra{-1} S_- \ket{1} & \bra{-1} S_- \ket{0} & \bra{-1} S_- \ket{-1}
\end{pmatrix}
=
\begin{pmatrix}
  0 & 0 & 0 \\
  \sqrt{2}\hbar & 0 & 0 \\
  0 & \sqrt{2}\hbar & 0
\end{pmatrix}
\]

Note duas coisas: primeiramente, só precisamos calcular uma dentre $S_+$ e $S_-$ para obter a outra, pois, como $S_- = S_+^\dagger$, podemos obter a representação matricial de uma a partir da outra simplemente transpondo e conjugando. Ademais, note que $S_+$ possui apenas a diagonal imediatamente acima da diagonal principal preenchida (e $S_-$ a diagonal imediatamente abaixo). Isto sempre acontece com os operadores levantamento e abaixamento (afinal, eles são operadores de levantamento e abaixamento!).

Agora, é imediato obter

\[
S_x = \frac{S_+ + S_-}{2} = \begin{pmatrix}
  0 & \frac{\hbar}{\sqrt{2}} & 0 \\
  \frac{\hbar}{\sqrt{2}} & 0 & \frac{\hbar}{\sqrt{2}} \\
  0 & \frac{\hbar}{\sqrt{2}} & 0
\end{pmatrix}
\]

\noindent e

\[
S_y = \frac{S_+ - S_-}{2i} = \begin{pmatrix}
  0 & \frac{-i\hbar}{\sqrt{2}} & 0 \\
  \frac{i\hbar}{\sqrt{2}} & 0 & \frac{-i\hbar}{\sqrt{2}} \\
  0 & \frac{i\hbar}{\sqrt{2}} & 0
\end{pmatrix}
\]

Note que $S_x$ e $S_y$ são hermiteanos, como necessariamente deveriam ser!


% Exercício 2
\addcontentsline{toc}{section}{Exercício 2}
\item Cohen, página 476, Complemento $J_{IV}$, exercício 1.\newline
\textit{Precessão de Larmor}\newline
Considere uma partícula de spin $1/2$ de momento magnético $\vec{M} = \gamma \vec{S}$. Os estados de spin são descritos na base de vetores $\ket{+}$ e $\ket{-}$, autovetores de $S_z$ com autovalores $\hbar/2$ e $-\hbar/2$. No tempo $t=0$, o estado do sistema é

\begin{equation}
  \ket{\Psi(t=0)} = \ket{+}
\label{eq:psi.t0}
\end{equation}

\textit{Observação: na notação do Cohen $\ket{+}$ é o estado de spin up na direção z, ou sendo mais preciso, $\ket{1/2\,1/2}_z$.}\newline
\textit{Observação: na notação do Griffiths, $\gamma = \frac{qg_s}{2m}$, onde $q$ é a carga, $g_s$ é o fator giromagnético e $m$ é a massa da partícula.}

\begin{enumerate}[(a)]
 \item Se o observável $S_x$ é medido no tempo $t=0$, quais resultados são possíveis e com quais probabilidades?
 \item Em vez de medir o resultado no tempo $t=0$, o sistema é deixado evoluir sob a influência de um campo magnético na direção $y$, de módulo $B_0$. Calcule na base $\ket{+}$, $\ket{-}$, o estado do sistema no instante $t$.
 \item No tempo $t$, nós medimos os observáveis $S_x$, $S_y$ e $S_z$. Quais são os valores destas quantidades que podemos encontrar e com quais probabilidades?
 \item Qual é a relação entre $B_0$ e $t$ para qual o resultado de uma das medidas é sem incerteza? Dê uma interpretação física desta condição.
\end{enumerate}

\textbf{Resolução:}
  
%FIM DOS EXERCÍCIOS
\end{enumerate}
\end{document}
