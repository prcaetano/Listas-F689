\documentclass[a4paper, 12pt, notitlepage]{article}
\usepackage[brazil]{babel}
\usepackage[utf8]{inputenc}
\usepackage[hmargin=2cm,vmargin=3cm,bmargin=3cm]{geometry}
\usepackage{enumerate}
\usepackage{graphicx}
\usepackage{mathtools}
\usepackage{physics}
\usepackage{amsmath,amssymb,amsthm}  %pacotes para matemática, opções de indentação, e links
\usepackage{caption}  % caption to minipages
\usepackage{indentfirst}
\usepackage{makeidx}
\usepackage{hyperref}
\hypersetup{colorlinks=false}
\usepackage[T1]{fontenc}
\usepackage{microtype}

\usepackage[sc,osf]{mathpazo}   % With old-style figures and real smallcaps.
\linespread{1.030}              % Palatino leads a little more leading

% Euler for math and numbers
\usepackage[euler-digits,small]{eulervm}
\AtBeginDocument{\renewcommand{\hbar}{\hslash}}

% Latex plots and drawings
\usepackage{tikz}
\usetikzlibrary{arrows.meta, angles, quotes}
\tikzset{>={Latex[width=3mm,length=3mm]}}  %setas mais visíveis no tikz
\usepackage{pgfplots}
\usepgfplotslibrary{fillbetween}

% some useful math shortcuts
\newtheorem{lema}{Lema }
\newtheorem{teorema}{Teorema }
\newtheorem{corolario}[teorema]{Corolário }
\newtheorem{definicao}{Definição }[section]
\newtheorem{postulado}{Postulado }[section]
\newtheorem{proposicao}{Proposição }[section]
\newtheorem{problema}{Problema }
\newcommand{\cart}{\times}
\newcommand{\ses}{\Longleftrightarrow}
\newcommand{\entao}{\Longrightarrow}
\newcommand{\e}{\wedge}
\newcommand{\ou}{\vee}
\newcommand{\vazio}{\varnothing}
\newcommand{\sobre}{\longrightarrow}
\newcommand{\N}{\mathbb{N}}
\newcommand{\Q}{\mathbb{Q}}
\newcommand{\R}{\mathbb{R}}
\newcommand{\Z}{\mathbb{Z}}
\newcommand{\La}{\mathcal{L}}
\newcommand{\cmod}[3]{#1 \equiv #2\textrm{ (mod }#3\textrm{)}}
\newcommand{\tq}{\textrm{ tal que }}
\renewcommand{\qedsymbol}{$\blacksquare$}
\newcommand{\dsum}{\displaystyle \sum}
\newcommand{\divg}[1]{\vec{\nabla} \cdot #1}
\newcommand{\rot}[1]{\vec{\nabla} \times #1}
\newcommand{\vecb}[1]{\mathbf{ #1}}
\newcommand{\veb}[1]{\mathbf{\hat{#1}}}


\begin{document}
\title{Resolução da Prova 1 de Mecânica Quântica I\\ (F689, Turma B, 1o. semestre de 2015)}
\author{Pedro Rangel Caetano\footnote{Email: p.r.caetano@gmail.com}} 
\date{Universidade Estadual de Campinas, 1o. semestre de 2017}
\maketitle

%\tableofcontents
%\pagebreak

\begin{enumerate}

% Problema 1
\addcontentsline{toc}{section}{Problema 1}
\item A função de onda de uma partícula de massa $m$ é

\begin{equation*}
\psi(x, t=0) =
  \begin{cases}
    Ax & 0 < x < a/2 \\
    A(a-x) & a/2 < x < a
  \end{cases}
\end{equation*}

\begin{enumerate}
  \item Esboce $\psi(x, 0)$ e determine a constante $A$.
  \item Escreva a função de onda inicial em termos das funções discretas de energia do poço infinito e encontre a função $\psi(x,t)$.
  \item Qual é a probabilidade $P(E_1)$ de a medida de energia produza o valor $E_1$, o valor mais baixo possível de energia do sistema?
\end{enumerate}

O poço infinito tem soluções $\Psi_n(x) = \sqrt{\frac{2}{a}}\sin\left(\frac{n\pi x}{a}\right)\qquad E_n=\frac{n^2\hbar^2\pi^2}{2ma^2}$.

\begin{align*}
\int_0^{a/2} dx x \sin\left(\frac{n\pi x}{a}\right) &= \frac{a^2}{n^2\pi^2}\left(-n\pi\cos(n\pi/2) + \sin(n\pi/2)\right) \\
\int_0^a dx x^2 \sin\left(\frac{n\pi x}{a}\right) &= \frac{a^3}{n^3 \pi^3}\left(-2 + (2 - n^2\pi^2)\cos(n\pi) + 2n\pi\sin(n\pi)\right)
\end{align*}

\textbf{Resolução: }

\begin{enumerate}
  \item Temos
  
  \begin{align*}
  \int_{-\infty}^{\infty} |\psi(x,0)|^2 dx &= \int_0^{a/2} A^2 x^2 dx + \int_{a/2}^a A^2 (a-x)^2 dx \\
  &= A^2 \left[ \frac{a^3}{24} + \int_{a/2}^{0} x^2 (-dx) \right] \\
  &= A^2 \left[ \frac{a^3}{24} + \frac{a^3}{24} \right] \\
  &= A^2 \frac{a^3}{12}
  \end{align*}
  
  Portanto
  
  \begin{align*}
  \int_{-\infty}^{\infty} |\psi(x,0)|^2 dx &= 1 \\
  \Rightarrow A &= \sqrt{\frac{12}{a^3}}
  \end{align*}
  
  O esboço de $\psi(x,0)$ pode ser visto na Figura \ref{fig:psi.prob.1}.

  \begin{figure}[!h]
    \centering
    \begin{tikzpicture}[>=latex]
		\begin{axis}[
		  samples=500,
		  axis x line=center,
		  axis y line=center,
		  xtick={3, 6},
		  xticklabels={$a/2$, $a$},
		  ytick={1},
		  yticklabels={$Aa/2$},
		  xlabel={$x$},
		  ylabel={$\Psi(x)$},
		  xlabel style={below right},
		  ylabel style={above left},
		  xmin=-1,
		  xmax=7,
		  ymin=-0.5,
		  ymax=1.5]
		\addplot [mark=none,domain=-1:8] {x < 0 ? 0 : x <= 3 ? x/3 : (x <= 6 ? (6 - x)/(6 - 3) : 0)};
		\end{axis}
	\end{tikzpicture}
  \caption{Gráfico de $\Psi(x)$.}
  \label{fig:psi.prob.1}
  \end{figure}
  
  \item Temos
  
  \begin{equation*}
  \psi(x,0) = \sum_{n = 1}^{\infty} c_n \psi_n(x)
  \end{equation*}
  
  \noindent com
  
  \begin{equation*}
  c_n = \sqrt{\frac{2}{a}} \int_0^a \sin\left(\frac{n\pi x}{a}\right) \psi(x,0) dx
  \end{equation*}
  
  Calculando os coeficientes $c_n$:
  
  \begin{align*}
  c_n &= \frac{2\sqrt{6}}{a^2}\left[ \int_0^{a/2} x \sin\left(\frac{n\pi x}{a}\right) dx + \int_{a/2}^a (a-x)\sin\left(\frac{n\pi x}{a}\right) \right] \\
  &= \frac{2\sqrt{6}}{a^2} \left[ \int_0^{a/2} x \sin\left(\frac{n\pi x}{a}\right) dx  + \int_{a/2}^0 x \sin\left(n\pi - \frac{n\pi x}{a}\right) (-dx) \right] \\
  &= \frac{2\sqrt{6}}{a^2} \left[ \int_0^{a/2} x \sin\left(\frac{n\pi x}{a}\right) dx - \int_0^{a/2} \sin\left(\frac{n\pi x}{a}\right) \cos(n\pi) dx \right] \\
  &= \frac{2\sqrt{6}}{a^2}\left(1 - \cos(n\pi)\right) \int_0^{a/2} x \sin\left(\frac{n\pi x}{a}\right) dx \\
  &= \frac{2\sqrt{6}}{a^2}\left(1 - \cos(n\pi)\right) \frac{a^2}{n^2 \pi^2} \left(-n\pi \cos(n\pi/2) + \sin(n\pi/2)\right) \\
  &= 
  \begin{cases}
  0 &\text{ para $n$ par} \\
  \frac{4\sqrt{6}}{n^2 \pi^2} \sin\left(\frac{n\pi}{2}\right) &\text{ para $n$ ímpar}
  \end{cases}
  \end{align*}
  
  Portanto
  
  \begin{align*}
  \psi(x,t) &= \sum_{\text{n ímpar}} c_n \psi_n(x) e^{-i\omega_n t}
  \end{align*}
  
  \noindent onde $\omega_n = E_n/\hbar = n^2\pi^2\hbar/(2ma^2)$.
  
  \item A probabilidade de obter no instante $t$ o valor $E_n$ para  a energia é dada por
  
  \begin{align*}
  P(E = E_n) &= \left|\int \psi_n^{\ast}(x) \psi(x, t) dx\right|^2 \\
  &= \left|c_n e^{-i\omega_n t}\right|^2 \\
  &= \left|c_n\right|^2
  \end{align*}
  
  Então,
  
  \begin{align*}
  P(E = E_1) &= \left|\frac{4\sqrt{6}}{\pi^2}\right|^2 \\
  &= \frac{96}{\pi^4}
  \end{align*}

\end{enumerate}

% Problema 2
\addcontentsline{toc}{section}{Problema 2}
\item Seja o potencial $V(x)$ dado por

\begin{equation*}
  V(x) = 
  \begin{cases}
  0 & x < 0 \\
  V_1 & 0 < x < a \\
  V_2 & a < x < b \\
  V_3 & b < x < c \\
  0 & x > c
  \end{cases}
\end{equation*}

\noindent onde $V_1 > V_3 > V_2$. Observe o ordenamento dos valores do potencial.

\begin{enumerate}
  \item Assuma que a energia da partícula $V_2 < E < V_3$. Determine as soluções nas diferentes regiões. As soluções encontradas tem estados ligados?
  \item Quais são as regiões classicamente proibidas no caso $V_3 < E < V_2$. Existem soluções quanticamente permitidas nesta região?
  \item Mostre graficamente uma possível solução $\psi(x)$ e seu comportamento em todas as regiões.
\end{enumerate}

\textbf{Resolução: }
\begin{enumerate}
  \item Sabemos que as soluções da equação de Schrödinger em regiões de potencial $V$ constante são
  \begin{itemize}
    \item ondas evanescentes da forma $e^{\pm k x}$, com $k = \frac{\sqrt{2m(V - E)}}{\hbar}$, se $V < E$ ou
    \item ondas planas da forma $e^{ikx}$, com $k = \frac{\sqrt{2m(E-V)}}{\hbar}$, se $V > E$.
  \end{itemize}
  Tendo em vista que $V_1 > V_3 > E > V_2$ as soluções serão ondas planas para $x < 0$, $a < x < b$ e $x > c$ e ondas evanescentes para $0 < x < a$ e $b < x < c$. Assumindo que a partícula vem da esquerda para a direita:
  
  \begin{equation*}
  \psi(x) = 
    \begin{cases}
      A e^{ik_0x} & x < 0 \\
      B e^{-k_1x} + Ce^{k_1 x} & 0 < x < a \\
      D e^{ik_2x} + E e^{-ik_3 x} & a < x < b \\
      F e^{-k_3x} + G e^{k_3 x} & b < x < c \\
      H e^{ik_0x} & x > c
    \end{cases}
  \end{equation*}
  
  \noindent onde as constantes de $A$ a $H$ são fixadas pelas condições de continuidade de $\psi$ e da derivada de $\psi$ e pela normalização, e $k_i = \frac{\sqrt{2m|E - V_i|}}{\hbar}$ para $i \neq 0$ e $k_0 = \frac{\sqrt{2mE}}{\hbar}$.
  
  Não há estados ligados: a condição para isto é que $E < 0$ (já que $\lim_{x \to \pm \infty} V(x) = 0$), mas não existem estados que satisfaçam isso, já que neste caso $E < V_{min}$ e portanto as soluções não são normalizáveis (da equação de Schrödinger obtemos que a derivada de $\psi$ e $\psi$ deveriam ter o mesmo sinal para todo valor de $x$ caso isso ocorresse, mas neste caso o gráfico da função se afasta do eixo $x$ e disto decorre que é impossível que a integral de $|\psi|^2$ convirja).
  
  \item Esta condição é impossível pois $V_3 > V_2$. As regiões classicamente proibidas são, porém, as regiões para as quais $E < V$. Se $E < V_2$, a região $a < x < b$ é classicamente proibida.
  
  \item Confira a Figura \ref{fig:ex2}.
  
  \begin{figure}[h!]
    \centering
    \begin{tikzpicture}[>=latex]
	    \begin{axis}[
	      samples=500,
	      axis x line=center,
	      axis y line=center,
        xlabel={$x$},
	      ylabel={$y$},
	      xlabel style={below right},
        ylabel style={below left},
        ytick=\empty,
        xtick={0, 1, 2, 3},
        xticklabels={$0$, $a$, $b$, $c$},
	      xmin=-3,
	      xmax=4,
  	    ymin=-3.5,
		    ymax=3.5]
		    \addplot [mark=none,domain=-3:0] {2*cos(1800*x)} ;
		    \addplot [mark=none,domain=0:1] {2*exp(-1*x)} ;
		    \addplot [mark=none,domain=1:2] {2*exp(-1)*cos(415*pi*(x-1))};
		    \addplot [mark=none,domain=2:3] {2*exp(-1)*cos(415*pi*1)*exp(-0.5*(x-2))};
		    \addplot [mark=none,domain=3:6] {2*exp(-1)*cos(415*pi)*exp(-0.5)*cos(1800*(x-3))};
        \addplot [mark=none,color=blue,domain=-3:6, ultra thick] {x<0? 0 : (x < 1? 2.0 : (x < 2? 1.0 : (x < 3? 1.5 : 0)))} ;
		  \end{axis}
    \end{tikzpicture}
    \caption{Esboço da função de onda no caso em que a partícula incide da esquerda para a direita.}
    \label{fig:ex2}
  \end{figure}
\end{enumerate}

% Problema 3
\addcontentsline{toc}{section}{Problema 3}
\item Seja a Equação de Schrödinger independente do tempo.

\begin{enumerate}
  \item Escreve a Equação de Schrödinger independente do tempo para o caso unidimensional. Quais são as condições gerais que qualquer função de onda deve satisfazer?
  \item Assuma que o potencial unidimensional $V(x)$ seja dado por $V(x) = -\alpha\left(\delta(x+a) + \delta(x-a)\right)$. Ache a solução geral da função de onda devido a este potencial quando a energia for $E < 0$.
  \item Encontre a condição do estado ligado neste caso. Não é necessário resolver a equação.
\end{enumerate}
A função Delta de Dirac satisfaz $\int f(x)\delta(x-a)dx = f(a)$.

\textbf{Resolução: }
Confira o exercício 5 da lista 2.

%FIM DOS PROBLEMAS
\end{enumerate}
\end{document}
